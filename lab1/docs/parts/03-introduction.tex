\chapter*{Введение}
\addcontentsline{toc}{chapter}{Введение}

В данной лабораторной работе будет рассмотрено расстояние Левенштейна.
Данное расстояние показывает минимальное количество операций (вставка, удаление, замена), которое необходимо для преобразования одной строки в другую. Это расстояние помогает определить схожесть двух строк.

Впервые задачу поставил в 1965 году советский математик Владимир Левенштейн при изучении последовательностей 0 -- 1, впоследствии более общую задачу для произвольного алфавита связали с его именем.

Расстояние Левенштейна применяется в теории информации и компьютерной лингвистике для решения следующих задач:
\begin{itemize}[label=---]
	\item исправление ошибок в слове(в поисковых системах, базах данных, при вводе текста, при автоматическом распознавании отсканированного текста или речи);
	\item сравнение текстовых файлов утилитой diff;
	\item для сравнения геномов, хромосом и белков в биоинформатике.
\end{itemize}

Метод динамического программирования был предложен и обоснован Р. Беллманом в начале 1960-х годов~\cite{ulianov}.
Первоначально метод создавался в целях существенного сокращения перебора для решения целого ряда задач экономического характера, формулируемых в терминах задач целочисленного программирования.
Однако Р. Беллман и Р. Дрейфус показали, что он применим к достаточно широкому кругу задач, в том числе к задачам поиска расстояния Левенштейна и Дамерау-Левенштейна.

Целью данной лабораторной работы является изучение алгоритмов поиска расстояний Левенштейна и Дамерау-Левенштейна.

Для поставленной цели необходимо выполнить следующие задачи.
\begin{enumerate}[label={\arabic*)}]
	\item Описать алгоритмы поиска расстояний Левенштейна и Дамерау-Ле-

	венштейна.
	\item Создать программное обеспечение, реализующее следующие алгоритмы:
	\begin{itemize}[label=---]
		\item нерекурсивный алгоритм поиска расстояния Левенштейна;
		\item нерекурсивный алгоритм поиска расстояния Дамерау-Левен-

		штейна;
		\item рекурсивный алгоритм поиска расстояния Дамерау-Левенштей-

		на;
		\item рекурсивный алгоритм поиска расстояния Дамерау-Левенштей-

		на с кэшированием.
	\end{itemize}
	\item Выбрать инструменты для замера процессорного времени необходимого для выполнения реализаций алгоритмов.
	\item Провести анализ затрат реализаций алгоритмов по времени и по памяти, определить влияющие на них факторы.
\end{enumerate}
