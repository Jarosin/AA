\chapter*{Заключение}
\addcontentsline{toc}{chapter}{Заключение}

В результате исследования было определено, что время работы реализаций алгоритмов нахождения расстояний Левенштейна и Дамерау-Левенштейна растет в геометрической прогрессии при увеличении длин строк.
Лучшие показатели по времени дает нерекурсивная реализация алгоритма нахождения расстояния Дамерау-Левенштейна и его рекурсивная реализация с кэшированием, использование которых приводит к 21-кратному превосходству по времени работы уже на длине строки в 4 символа за счет сохранения необходимых промежуточных вычислений.

В результате оценки лучшим по памяти является рекурсивный алгоритм поиска расстояния Дамерау-Левенштейна.
Нерекурсивные алгоритмы дали одинаковые результаты, используя памяти меньше, чем рекурсивный алгоритм.
Худшей по памяти оказался рекурсивный алгоритм поиска расстояния Дамерау-Левенштейна с кешированием.

Цель данной лабораторной работы была достигнута --- были изучены алгоритмы поиска расстояний Левенштейна и Дамерау-Левенштейна.

Для достижения поставленных целей были выполнены все задачи.
\begin{enumerate}[label={\arabic*)}]
	\item Описаны алгоритмы поиска расстояния Левенштейна и \newline Дамерау-Левенштейна;
	\item Создано программное обеспечение, реализующее следующие алгоритмы:
	\begin{itemize}[label=---]
		\item нерекурсивный метод поиска расстояния Левенштейна;
		\item нерекурсивный метод поиска расстояния Дамерау-Левенштейна;
		\item рекурсивный метод поиска расстояния Дамерау-Левенштейна;
		\item рекурсивный поиска расстояния Дамерау-Левенштейна с кэшированием.
	\end{itemize}
	\item Выбраны инструменты для замера процессорного времени требуемого для выполнения реализаций алгоритмов.
	\item Проведен анализ затрат работы программы по времени и по памяти, определены влияющие на них факторы.
\end{enumerate}
