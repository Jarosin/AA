\chapter{Технологическая часть}

В данном разделе будут приведены требования к программному обеспечению, средства реализации, листинг кода и функциональные тесты.

\section{Средства реализации}

Для реализации данной лабораторной работы был выбран язык $C++$ ~\cite{cpp-lang}. Данный выбор был сделан, так как у языка есть встроенный модуль $ctime$ для измерения процессорного времени и тип данных, позволяющий хранить как кириллические символы, так и латинские --- $std::wstring$, что необходимо согласно третьему требованию из п.\ref{section:requirements} и соответствуют выдвинутым техническим требованиям.

Время работы было замерено с помощью функции \textit{clock()} из библиотеки \textit{сtime}~\cite{cpp-lang-time}.

\clearpage

\section{Сведения о модулях программы}

Данная программа разбита на следующие модули:

\begin{itemize}
	\item \texttt{main.cpp} --- файл, содержащий точку входа в программу, из которой происходит вызов алгоритмов по разработанному интерфейсу;
	\item \texttt{levenshtein.cpp} --- файл содержит функции поиска расстояния Левенштейна и Дамерау-Левенштейна;
	\item \texttt{memory.cpp} --- файл содержит функции динамического выделения и очищения памяти для матрицы;
	\item \texttt{time.cpp} --- файл содержит функции, замеряющее процессорное время алгоритмов поиска расстояния Левенштейна и Дамерау-Левен-штейна;
\end{itemize}

\section{Реализация алгоритмов}

В листингах \ref{lst:lev_mtr} -- \ref{lst:dameray_lev_rec_hash} приведены реализации алгоритмов поиска расстояния Левенштейна (только нерекурсивный алгоритм) и Дамерау-Левен-штейна (нерекурсивный, рекурсивный и рекурсивный с кэшированием).
В листинге \ref{lst:allocate_mtr} приведены реализации алгоритмов выделения памяти под матрицу и очищения памяти из под нее.

\clearpage

\begin{lstlisting}[label=lst:lev_mtr,caption=Функция нахождения расстояния Левенштейна с использованием матрицы]
int levenshtein_matrix(std::wstring &first_word, std::wstring &second_word)
{
	int first_length = first_word.length();
	int second_length = second_word.length();
	int **matrix = alloc_matrix(first_length + 1, second_length + 1);
	for (int i = 0; i < first_length + 1; i++)
	{
		for (int j = 0; j < second_length + 1; j++)
		{
			if (i == 0 || j == 0)
			{
				matrix[i][j] = abs(i - j);
			}
			else
			{
				int equal = 1;
				if (first_word[i - 1] == second_word[j - 1])
					equal = 0;
				matrix[i][j] = std::min({matrix[i - 1][j] + 1,
											matrix[i][j - 1] + 1,
											matrix[i - 1][j - 1] + equal});
			}
		}
	}
	int res = matrix[first_length][second_length];
	free_matrix(matrix, first_length + 1);
	return res;
}
\end{lstlisting}

\clearpage

\begin{lstlisting}[label=lst:dameray_lev_rec,caption=Функция нахождения расстояния Дамерау-Левенштейна с использованием матрицы]

int damerau_levenshtein_matrix(std::wstring &first_word,
	std::wstring &second_word)
{
	int first_length = first_word.length();
	int second_length = second_word.length();
	int **matrix = alloc_matrix(first_length + 1, second_length + 1);
	for (int i = 0; i < first_length + 1; i++)
	{
		for (int j = 0; j < second_length + 1; j++)
		{
			if (i == 0 || j == 0)
			{
				matrix[i][j] = abs(i - j);
			}
			else
			{
				int equal = 1;
				if (first_word[i - 1] == second_word[j - 1])
				equal = 0;
				matrix[i][j] = std::min({matrix[i - 1][j] + 1,
							matrix[i][j - 1] + 1,
							matrix[i - 1][j - 1] + equal});
				if (i > 1 && j > 1 && first_word[i - 2] == second_word[j - 1] &&
					first_word[i - 1] == second_word[j - 2])
				{
					matrix[i][j] = std::min(matrix[i][j],
								matrix[i - 2][j - 2] + 1);
				}
			}
		}
	}
	int res = matrix[first_length][second_length];
	free_matrix(matrix, first_length + 1);
	return res;
}
\end{lstlisting}

\clearpage

\begin{lstlisting}[label=lst:dameray_lev_mtr,caption=Функция нахождения расстояния Дамерау-Левенштейна рекурсивно]
int damerau_lev_rec(std::wstring &first_word, std::wstring &second_word,
	int n, int m)
{
	if (n == 0 || m == 0)
		return abs(n - m);

	int equal = 1;
	if (first_word[n - 1] == second_word[m - 1])
		equal = 0;

	int res = std::min({damerau_lev_rec(first_word, second_word, n, m - 1) + 1,
						damerau_lev_rec(first_word, second_word, n - 1, m) + 1,
						damerau_lev_rec(first_word, second_word,
						n - 1, m - 1) + equal});

	if (n > 1 && m > 1 && first_word[n - 1] == second_word[m - 2] &&
		first_word[n - 2] == second_word[m - 1])
	{
		res = std::min(res, damerau_lev_rec(first_word, second_word,
											n - 2, m - 2) + 1);
	}

	return res;
}

int damerau_lev_recursion(std::wstring &first_word, std::wstring &second_word)
{
    return damerau_lev_rec(first_word, second_word,
                           first_word.length(), second_word.length());
}
\end{lstlisting}

\clearpage

\begin{lstlisting}[label=lst:dameray_lev_rec_hash,caption=Функция нахождения расстояния Дамерау-Левенштейна рекурсивно c кешированием]

int damerau_lev_rec_hash(std::wstring &first_word, std::wstring &second_word,
                    int **matrix, int n, int m)
{
    if (matrix[n][m] != -1) {
        return matrix[n][m];
    }

    if (n == 0 || m == 0) {
        matrix[n][m] = std::max(m, n);
        return matrix[n][m];
    }

    int equal = 1;
    if (first_word[n - 1] == second_word[m - 1])
        equal = 0;

    matrix[n][m] = std::min({damerau_lev_rec_hash(first_word, second_word, matrix, n, m - 1) + 1,
                       damerau_lev_rec_hash(first_word, second_word, matrix, n - 1, m) + 1,
                       damerau_lev_rec_hash(first_word, second_word, matrix,
                                       n - 1, m - 1) + equal});

    if (n > 1 && m > 1 && first_word[n - 1] == second_word[m - 2] &&
        first_word[n - 2] == second_word[m - 1])
    {
        matrix[n][m] = std::min(matrix[n][m], damerau_lev_rec_hash(first_word, second_word, matrix,
                                            n - 2, m - 2) + 1);
    }

    return matrix[n][m];
}

int damerau_lev_recursion_hash(std::wstring &first_word, std::wstring &second_word)
{
    int first_length = first_word.length();
    int second_length = second_word.length();

    int **matrix = alloc_matrix(first_length + 1, second_length + 1);
    for (int i = 0; i < first_length + 1; i++) {
        for (int j = 0; j < second_length + 1; j++) {
            matrix[i][j] = -1;
        }
    }

    int res = damerau_lev_rec_hash(first_word, second_word, matrix,
                           first_word.length(), second_word.length());

    free_matrix(matrix, first_length + 1);
    return res;
}

\end{lstlisting}

\clearpage

\begin{lstlisting}[label=lst:allocate_mtr,caption=Функции динамического выделения и очищения памяти под матрицу]
int **alloc_matrix(int n, int m) {
    if (n < 1 || m < 1) {
        return nullptr;
    }
    int **matrix = new int*[n];
    for (int i = 0; i < n; i++) {
        matrix[i] = new int[m];
    }
    return matrix;
}

int free_matrix(int **matrix, int n) {
    if (n < 1 || matrix == nullptr) {
        return 1;
    }
    for (int i = 0; i < n; i++) {
        delete[] matrix[i];
    }
    delete[] matrix;
    return 0;
}
\end{lstlisting}

\clearpage

\section{Функциональные тесты}

В таблице \ref{tbl:func_tests} приведены функциональные тесты для алгоритмов вычисления расстояний Левенштейна и Дамерау—Левенштейна.
Все тесты пройдены успешно.

\begin{table}[ht]
	\small
	\begin{center}
		\begin{threeparttable}
		\caption{Функциональные тесты}
		\label{tbl:func_tests}
		\begin{tabular}{|c|c|r|r|r|r|}
			\hline
			\multicolumn{2}{|c|}{\bfseries Входные данные}
			& \multicolumn{4}{c|}{\bfseries Расстояние и алгоритм} \\
			\hline
			&
			& \multicolumn{1}{c|}{\bfseries Левенштейна}
			& \multicolumn{3}{c|}{\bfseries Дамерау-Левенштейна} \\ \cline{3-6}

			\bfseries Строка 1 & \bfseries Строка 2 & \bfseries Итеративный & \bfseries Итеративный

			& \multicolumn{2}{c|}{\bfseries Рекурсивный} \\ \cline{5-6}
			& & & & \bfseries Без кеша & \bfseries С кешом \\
			\hline
			a & b & 1 & 1 & 1 & 1 \\
			\hline
			a & a & 0 & 0 & 0 & 0 \\
			\hline
			12 & 123 & 1 & 1 & 1 & 1 \\
			\hline
			123 & 12 & 1 & 1 & 1 & 1 \\
			\hline
			hello & helol & 2 & 1 & 1 & 1 \\
			\hline
			привет & привте & 2 & 1 & 1 & 1 \\
			\hline
			слон & слоны & 1 & 1 & 1 & 1 \\
			\hline
		\end{tabular}
		\end{threeparttable}
	\end{center}
\end{table}

\section*{Вывод}

Были реализованы алгоритмы поиска расстояния Левенштейна итеративно, а также поиска расстояния Дамерау–Левенштейна итеративно, рекурсивно и рекурсивного с кеширования.
Проведено тестирование реализаций алгоритмов.
