\chapter{Технологическая часть}

В данном разделе рассмотрены средства реализации, а также представлены листинги реализаций рассматриваемых алгоритмов.

\section{Средства реализации}

В данной работе для реализации был выбран язык программирования $Python$ \cite{python-lang}.
В текущей лабораторной работе требуется замерить процессорное время работы выполняемой программы.
Инструменты для этого присутствуют в выбранном языке программирования.

Время работы было замерено с помощью функции \textit{process\_time(...)} из библиотеки $time$ \cite{python-lang-time}.


\section{Сведения о файлах программы}

Данная программа разбита на следующие файлы:
\begin{itemize}
	\item $main.py$ --- файл, содержащий точку входа;
	\item $menu.py$ --- файл, содержащий код меню программы;
	\item $utils.py$ --- файл, содержащий служебные алгоритмы;
	\item $constants.py$ --- файл, содержащий константы программы;
	\item $algorythms.py$ --- файл, содержащий код всех алгоритмов.
\end{itemize}

\section{Реализация алгоритмов}

В листинге \ref{lst:full_comb} представлен реализация алгоритм полного перебора путей, а в листингах \ref{lst:ant}--\ref{lst:update} --- муравьиный алгоритм и дополнительные к нему функции.


\lstinputlisting[label=lst:full_comb,caption=Реализаиция алгоритма полного перебора, firstline=6,lastline=27]{../src/algorithms.py}

\clearpage

\lstinputlisting[label=lst:ant,caption=Реализаиция муравьиного алгоритма, firstline=120,lastline=147]{../src/algorithms.py}

\clearpage

\lstinputlisting[label=lst:find-way,caption=Реализация алгоритма нахождения массива вероятностей переходов в непосещенные порты, firstline=91,lastline=107]{../src/algorithms.py}

\lstinputlisting[label=lst:calc-phero,caption=Реализация алгоритма нахождения массива вероятностей переходов в непосещенные порты, firstline=39,lastline=42]{../src/algorithms.py}

\lstinputlisting[label=lst:choose-next,caption=Реализация алгоритма выбора следующего порта, firstline=109,lastline=117]{../src/algorithms.py}

\clearpage

\lstinputlisting[label=lst:update,caption=Реализация алгоритма обновления матрицы феромонов, firstline=66,lastline=82]{../src/algorithms.py}

\clearpage

\section{Функциональные тесты}

В таблицах \ref{tbl:functional_test} -- \ref{tbl:functional_test_ants} приведены тесты для реализаций алгоритма полного перебора и муравьиного алгоритма соответственно.

Поскольку метод на основе муравьиного алгоритма не всегда находит кратчаший путь, тест будет считаться пройденным, если полученное значение отличается от эталонного не более, чем на 5.
Все функциональные тесты пройдены \textit{успешно}.

\begin{center}
	\captionsetup{justification=raggedright,singlelinecheck=off}
	\begin{longtable}[c]{|c|c|c|c|c|}
		\caption{Функциональные тесты для реализации алгоритма полного перебора\label{tbl:functional_test}} \\ \hline
		Матрица смежности & Ожидаемый результат & Результат программы \\
		\hline
		$ \begin{pmatrix}
			0 &  4 &  2 &  1 & 7 \\
			1 &  0 &  3 &  7 & 2 \\
			5 &  3 &  0 & 10 & 3 \\
			8 &  6 & 7 &  0 & 9 \\
			1 &  5 &  10 &  4 & 0
		\end{pmatrix}$ &
		11, [2, 1, 4, 0, 3] &
		11, [2, 1, 4, 0, 3] \\

		$ \begin{pmatrix}
			0 & 1 & 2 \\
			1 & 0 & 1 \\
			2 & 1 & 0
		\end{pmatrix}$ &
		3, [0, 1, 2] &
		3, [0, 1, 2] \\

		$ \begin{pmatrix}
			0 & 15 & 19 & 20 \\
			15 &  0 & 12 & 13 \\
			19 & 12 &  0 & 17 \\
			20 & 13 & 17 &  0
		\end{pmatrix}$ &
		44, [0, 1, 2, 3] &
		44, [0, 1, 2, 3] \\
		\hline
	\end{longtable}
\end{center}

\clearpage

\begin{center}
	\captionsetup{justification=raggedright,singlelinecheck=off}
	\begin{longtable}[c]{|c|c|c|c|c|}
		\caption{Функциональные тесты для реализации муравьиного алгоритма\label{tbl:functional_test_ants}} \\ \hline
		Матрица смежности & Ожидаемый результат & Результат программы \\
		\hline
		$ \begin{pmatrix}
			0 &  4 &  2 &  1 & 7 \\
			1 &  0 &  3 &  7 & 2 \\
			5 &  3 &  0 & 10 & 3 \\
			8 &  6 & 7 &  0 & 9 \\
			1 &  5 &  10 &  4 & 0
		\end{pmatrix}$ &
		11, [2, 1, 4, 0, 3] &
		11, [2, 1, 4, 0, 3] \\

		$ \begin{pmatrix}
			0 & 1 & 2 \\
			1 & 0 & 1 \\
			2 & 1 & 0
		\end{pmatrix}$ &
		3, [0, 1, 2] &
		3, [0, 1, 2] \\

		$ \begin{pmatrix}
			0 & 15 & 19 & 20 \\
			15 &  0 & 12 & 13 \\
			19 & 12 &  0 & 17 \\
			20 & 13 & 17 &  0
		\end{pmatrix}$ &
		44, [0, 1, 2, 3] &
		44, [0, 1, 2, 3] \\
		\hline
	\end{longtable}
\end{center}

\section*{Вывод}

Были представлены листинги всех реализаций алгоритмов --- полного перебора и муравьиного.
Также в данном разделе была приведена информации о выбранных средствах для разработки алгоритмов и сведения о файлах программы, проведено функциональное тестирование.
