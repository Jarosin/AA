\chapter*{Введение}
\addcontentsline{toc}{chapter}{Введение}

В данной лабораторной работе будут рассмотрены алгоритмы умножения матриц.
В программировании, как и в математике, на практике часто приходится прибегать к использованию матриц~\cite{book_matrix}.
Существует огромное количество областей их применения в этих сферах.
Матрицы активно используются при выводе различных формул в физике, например:
\begin{itemize}
	\item градиент;
	\item дивергенция;
	\item ротор.
\end{itemize}

Нельзя обойти стороной и различные операции над матрицами --- сложение, возведение в степень, умножение.
При различных задачах размеры матрицы могут достигать больших значений, поэтому оптимизация операций работы над матрицами является важной задачей в программировании.
В данной лабораторной работе пойдёт речь об оптимизациях операции умножения матриц.

Целью данной лабораторной работы является описание алгоритмов умножения матриц.

Для поставленной цели необходимо выполнить следующие задачи.
\begin{enumerate}[label={\arabic*)}]
	\item Рассмотреть три алгоритма умножения матриц.
	\item Создать программное обеспечение, реализующее следующие алгоритмы:
	\begin{itemize}[label=---]
		\item классический алгоритм умножения матриц;
		\item алгоритм Винограда;
		\item оптимизированный алгоритм Винограда.
	\end{itemize}
	\item Оценить трудоемкость алгоритмов умножения матриц.
	\item Провести анализ затрат работы программы по времени.
	\item Провести сравнительный анализ алгоритмов.
\end{enumerate}
