\chapter{Технологическая часть}

В данном разделе будут приведены средства реализации, листинги кода и функциональные тесты.

\section{Средства реализации}

Для написания реализации данной лабораторной работы был выбран язык программирования C++~\cite{cpp-lang}.
Данный выбор обусловлен наличием у языка встроенной библиотеки измерения процессорного времени и соответствием с выдвинутыми техническими требованиям.

Для измерения времени использовалось функция \textit{clock()} из модуля \textit{ctime}~\cite{cpp-lang-time}.

\clearpage

\section{Сведения о файлах программы}

Данная программа разбита на следующие файлы:

\begin{itemize}[label=---]
	\item \texttt{main.cpp} --- файл, содержащий точку входа в программу. В не происходит общение с пользователем и вызов алгоритмов;
	\item \texttt{matrix.cpp} --- файл содержит алгоритмы умножения матриц, вывод матриц;
	\item \texttt{io.cpp} --- файл содержит функции чтение матрицы;
	\item \texttt{cpu\_time.cpp} --- файл содержит функции, замеряющее процессорное время, затрачиваемое реализациями алгоритмов умножения матриц.
\end{itemize}

\section{Реализация алгоритмов}

В листингах \ref{lst:standart} -- \ref{lst:vinograd_opt} реализованы алгоритмы умножения матриц --- классический, алгоритм Винограда и оптимизированный алгоритм Винограда.

\lstinputlisting[label=lst:standart,caption={Функция, реализующая классический алгоритм умножения матриц}, firstline=89,lastline=105]{../src/matrix.cpp}

\lstinputlisting[label=lst:vinograd,caption={Функция, реализующая умножение матриц по алгоритму Винограда}, firstline=25,lastline=53]{../src/matrix.cpp}
\clearpage

\lstinputlisting[label=lst:vinograd_opt,caption={Функция, реализующая умножение матриц по оптимизированному алгоритму Винограда}, firstline=55,lastline=87]{../src/matrix.cpp}
\clearpage

\section{Функциональные тесты}

В таблице \ref{tbl:func_tests_std} и \ref{tbl:func_tests_vin} приведены функциональные тесты для разработанных реализаций алгоритмов умножения матриц. Все тесты пройдены успешно.

\begin{table}[ht]
	\small
	\begin{center}
		\begin{threeparttable}
		\caption{Функциональные тесты для классического алгоритма умножения матриц}
		\label{tbl:func_tests_std}
		\begin{tabular}{|c|c|c|c|c|}
			\hline
			\multicolumn{2}{|c|}{\bfseries Входные данные}
			& \multicolumn{2}{c|}{\bfseries Результат для классического алгоритма} \\
			\hline
			\bfseries Матрица 1
			& \bfseries Матрица 2
			& \bfseries Ожидаемый результат
			& \bfseries Фактический результат \\
			\hline
			$\begin{pmatrix}
				1 & 5 & 7\\
				2 & 6 & 8\\
				3 & 7 & 9
			\end{pmatrix}$
			&
			$\begin{pmatrix}
				&
			\end{pmatrix}$
			&
			\text{Сообщение об ошибке}
			&
			\text{Сообщение об ошибке} \\
			\hline
			$\begin{pmatrix}
				1 & 5 & 7\\
			\end{pmatrix}$
			&
			$\begin{pmatrix}
				1 & 2 & 3\\
			\end{pmatrix}$
			&
			\text{Сообщение об ошибке}
			&
			\text{Сообщение об ошибке} \\
			\hline
			$\begin{pmatrix}
				1 & 2 & 3\\
				4 & 5 & 6 \\
				7 & 8 & 9 \\
			\end{pmatrix}$
			&
			$\begin{pmatrix}
				1 & 0 & 0\\
				0 & 1 & 0 \\
				0 & 0 & 1 \\
			\end{pmatrix}$
			&
			$\begin{pmatrix}
				1 & 0 & 0\\
				0 & 1 & 0 \\
				0 & 0 & 1 \\
			\end{pmatrix}$
			&
			$\begin{pmatrix}
				1 & 2 & 3\\
				4 & 5 & 6 \\
				7 & 8 & 9 \\
			\end{pmatrix}$ \\
			\hline
			$\begin{pmatrix}
				3 & 5\\
				2 & 1\\
				9 & 7\\
			\end{pmatrix}$
			&
			$\begin{pmatrix}
				1 & 2 & 3\\
				4 & 5 & 6 \\
			\end{pmatrix}$
			&
			$\begin{pmatrix}
				23 & 31 & 39 \\
				6 & 9 & 12
			\end{pmatrix}$
			&
			$\begin{pmatrix}
				23 & 31 & 39 \\
				6 & 9 & 12
			\end{pmatrix}$ \\
			\hline
			$\begin{pmatrix}
				10
			\end{pmatrix}$
			&
			$\begin{pmatrix}
				35
			\end{pmatrix}$
			&
			$\begin{pmatrix}
				350
			\end{pmatrix}$
			&
			$\begin{pmatrix}
				350
			\end{pmatrix}$ \\
			\hline
		\end{tabular}
		\end{threeparttable}
	\end{center}
\end{table}

\begin{table}[ht]
	\small
	\begin{center}
		\begin{threeparttable}
		\caption{Функциональные тесты для умножения матриц по алгоритму Винограда}
		\label{tbl:func_tests_vin}
		\begin{tabular}{|c|c|c|c|c|}
			\hline
			\multicolumn{2}{|c|}{\bfseries Входные данные}
			& \multicolumn{2}{c|}{\bfseries Результат для алгоритма Винограда} \\
			\hline
			\bfseries Матрица 1
			& \bfseries Матрица 2
			& \bfseries Ожидаемый результат
			& \bfseries Фактический результат \\
			\hline
			$\begin{pmatrix}
				1 & 5 & 7\\
				2 & 6 & 8\\
				3 & 7 & 9
			\end{pmatrix}$
			&
			$\begin{pmatrix}
				&
			\end{pmatrix}$
			&
			\text{Сообщение об ошибке}
			&
			\text{Сообщение об ошибке} \\
			\hline
			$\begin{pmatrix}
				1 & 5 & 7\\
			\end{pmatrix}$
			&
			$\begin{pmatrix}
				1 & 2 & 3\\
			\end{pmatrix}$
			&
			\text{Сообщение об ошибке}
			&
			\text{Сообщение об ошибке} \\
			\hline
			$\begin{pmatrix}
				1 & 2 & 3\\
				4 & 5 & 6 \\
				7 & 8 & 9 \\
			\end{pmatrix}$
			&
			$\begin{pmatrix}
				1 & 0 & 0\\
				0 & 1 & 0 \\
				0 & 0 & 1 \\
			\end{pmatrix}$
			&
			$\begin{pmatrix}
				1 & 0 & 0\\
				0 & 1 & 0 \\
				0 & 0 & 1 \\
			\end{pmatrix}$
			&
			$\begin{pmatrix}
				1 & 2 & 3\\
				4 & 5 & 6 \\
				7 & 8 & 9 \\
			\end{pmatrix}$ \\
			\hline
			$\begin{pmatrix}
				3 & 5\\
				2 & 1\\
				9 & 7\\
			\end{pmatrix}$
			&
			$\begin{pmatrix}
				1 & 2 & 3\\
				4 & 5 & 6 \\
			\end{pmatrix}$
			&
			\text{Сообщение об ошибке}
			&
			\text{Сообщение об ошибке} \\
			\hline
			$\begin{pmatrix}
				10
			\end{pmatrix}$
			&
			$\begin{pmatrix}
				35
			\end{pmatrix}$
			&
			$\begin{pmatrix}
				350
			\end{pmatrix}$
			&
			$\begin{pmatrix}
				350
			\end{pmatrix}$ \\
			\hline
		\end{tabular}
		\end{threeparttable}
	\end{center}
\end{table}

\clearpage

\section*{Вывод}

Были представлены листинги реализаций всех алгоритмов умножения матриц --- стандартного, Винограда и оптимизированного Винограда.
Также в данном разделе была приведена информация о выбранных средствах для разработки реализаций алгоритмов.
