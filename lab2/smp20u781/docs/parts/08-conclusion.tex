\chapter*{Заключение}
\addcontentsline{toc}{chapter}{Заключение}

В результате исследования было определено, что реализация классического алгоритма умножения матриц проигрывает по времени реализации алгоритма Винограда примерно на 11\% при размерах матриц больше 10 из-за того, что в алгоритме Винограда часть вычислений происходит заранее, а также сокращается часть сложных операций --- операций умножения, поэтому предпочтение следует отдавать алгоритму Винограда.
Но лучшие показатели по времени выдает реализация оптимизированного алгоритма Винограда --- он примерно на 10\% быстрее реализации алгоритма Винограда на размерах матриц выше 10 из-за замены операций равно и плюс на операцию плюс-равно, а также за счёт замены операции умножения операцией сдвига, что дает проводить часть вычислений быстрее.

При выборе самого быстрого алгоритма предпочтение стоит отдавать оптимизированному алгоритму Винограда.
Также стоит упомянуть, что написанная реализация алгоритма Винограда работает на чётных размерах матриц примерно в 1.1 раза быстрее, чем на нечётных, что связано с тем, что нужно произвести часть дополнительных вычислений для крайних строк и столбцов матриц.

Цель, которая была поставлена в начале лабораторной работы, была достигнута.
В ходе выполнения лабораторной работы были решены все задачи.
\begin{enumerate}[label={\arabic*)}]
	\item Рассмотрены три алгоритма умножения матриц.
	\item Создано программное обеспечение, реализующее следующие алгоритмы:
	\begin{itemize}[label=---]
		\item классический алгоритм умножения матриц;
		\item алгоритм Винограда;
		\item оптимизированный алгоритм Винограда.
	\end{itemize}
	\item Оценены трудоемкости алгоритмов умножения матриц.
	\item Проведен анализ затрат работы программы по времени.
	\item Проведен сравнительный анализ алгоритмов.
\end{enumerate}
