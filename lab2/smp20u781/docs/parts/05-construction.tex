\chapter{Конструкторская часть}

В этом разделе будут представлены описания используемых типов данных, а также схемы алгоритмов перемножения матриц – Винограда и оптимизации алгоритма Винограда

\section{Требования к ПО}
К программе предъявлен ряд функциональных требований:
\begin{itemize}
	\item входные данные --- размеры матрицы, целые числа и матрица, двумерный массив целых чисел, выходные данные --- матрица;
	\item программа должна предоставлять интерфейс для выбора действий;
	\item программа должна давать возможность замерить процессорное время, затрачиваемое реализациями алгоритмов умножения матриц.
\end{itemize}

\section{Разработка алгоритмов}

На рисунке \ref{img:standart} представлена схема алгоритма стандартного умножения матриц.
На рисунках \ref{img:vinograd_alg_1} и \ref{img:vinograd_alg_2} --- схема алгоритма Винограда, а на \ref{img:vinograd_opt_alg_1} и \ref{img:vinograd_opt_alg_2} --- схема оптимизированного алгоритма Винограда.



\clearpage
\imgScale{0.70}{standart}{Схема стандартного алгоритма умножения матриц}
\clearpage
\imgScale{0.65}{vinograd_alg_1}{Схема умножения матриц по алгоритму Винограда (часть 1)}
\clearpage
\imgScale{0.65}{vinograd_alg_2}{Схема умножения матриц по алгоритму Винограда (часть 2)}
\clearpage
\imgScale{0.65}{vinograd_opt_alg_1}{Схема умножения матриц по оптимизированному алгоритму Винограда (часть 1)}
\clearpage
\imgScale{0.65}{vinograd_opt_alg_2}{Схема умножения матриц по оптимизированному алгоритму Винограда (часть 2)}
\clearpage

\section{Описание используемых типов данных}

При реализации алгоритмов будут использованы следующие структуры данных:

\begin{itemize}
	\item матрица --- двумерный список целых чисел;
	\item вектор --- одномерный список целых чисел.
\end{itemize}

\section{Модель для оценки трудоемкости}

Введем модель вычислений для определения трудоемкости каждого алгоритма умножения матриц.
\begin{enumerate}[label={\arabic*)}]
	\item Трудоемкость базовых операций:
	\begin{itemize}[label=---]
		\item для операций $+, -, =, +=, -=, ==, !=, <, >, <=, >=, [], ++,\\ {-}-,
			\&\&, >>, <<, ||, \&, |$
		трудоемкость равна 1;
		\item для операций $*, /, \%, *=, /=, \%=$
		трудоемкость равна 2.
	\end{itemize}
	\item Трудоемкость условного оператора:
	\begin{equation}
		\label{for:if}
		f_{if} = f_{\text{условия}} +
		\begin{cases}
			min(f_1, f_2), & \text{лучший случай},\\
			max(f_1, f_2), & \text{худший случай},
		\end{cases}
	\end{equation}
	где $f1$ и $f2$ --- трудоемкости соответствующих ветвей условного оператора.
	\item Трудоемкость цикла:
	\begin{equation}
		\label{for:for}
		\begin{gathered}
			f_{for} = f_{\text{инициализация}} + f_{\text{сравнения}} + M_{\text{итераций}} \cdot (f_{\text{тело}} +\\
			+ f_{\text{инкремент}} + f_{\text{сравнения}}).
		\end{gathered}
	\end{equation}
	\item Трудоемкость передачи параметра в функции и возврат из функции равны 0.
\end{enumerate}

\section{Трудоемкость алгоритмов}
\subsection{Классический алгоритм}

Для стандартного алгоритма умножения матриц трудоемкость будет слагаться из следующих компонент:

\begin{itemize}[label=---]
	\item внешнего цикла по $i \in [1 \ldots N]$ , трудоёмкость которого: $f = 2 + N \cdot (2 + f_{body})$;
	\item цикла по $j \in [1 \ldots T]$ , трудоёмкость которого: $f = 2 + 2 + T \cdot (2 + f_{body})$;
	\item цикла по $k \in [1 \ldots M]$ , трудоёмкость которого: $f = 2 + 2 + 14M$;
\end{itemize}

Поскольку трудоемкость стандартного алгоритма равна трудоемкости внешнего цикла, то она вычисляется по следующей формуле:
\begin{equation}
	\label{сomplexity:standart}
	\begin{gathered}
		f_{standart} = 2 + N \cdot (2 + 2 + T \cdot (2 + 2 + M \cdot (2 + 8 + 1 + 1 + 2)))= \\
		= 2 + 4N + 4NT + 14NMT \approx 14NMT = O(N^3).
	\end{gathered}
\end{equation}

\clearpage

\subsection{Алгоритм Винограда}

Чтобы вычислить трудоемкость алгоритма Винограда, необходимо рассчитать трудоемкость его частей.
\begin{enumerate}
	\item Создание и инициализация массивов $a_{tmp}$ и $b_{tmp}$, трудоёмкость которых вычисляется по формуле:
	\begin{equation}
		\label{сomplexity:v_init}
		f_{init} = N + M.
	\end{equation}
	\item Заполнение массива $a_{tmp}$, трудоёмкость которого вычисляется по формуле:
	\begin{equation}
		\label{сomplexity:v_atmp}
		\begin{gathered}
			f_{atmp} = 2 + N \cdot (4 + \frac{M}{2} \cdot (4 + 6 + 1 + 2 + 3 \cdot 2)) = \\
			= 2 + 4N + \frac{19NM}{2} = 2 + 4N + 9,5NM.
		\end{gathered}
	\end{equation}
	\item Заполнение массива $b_{tmp}$, трудоёмкость которого указана вычисляется по формуле:
	\begin{equation}
		\label{сomplexity:v_btmp}
		\begin{gathered}
			f_{btmp} = 2 + T \cdot (4 + \frac{M}{2} \cdot (4 + 6 + 1 + 2 + 3 \cdot 2)) = \\
			= 2 + 4T + \frac{19TM}{2} = 2 + 4T + 9,5TM.
		\end{gathered}
	\end{equation}
	\item Цикл заполнения для чётных размеров, трудоёмкость которого считается по формуле:
	\begin{equation}
		\label{сomplexity:v_cycle}
		\begin{gathered}
			f_{cycle} = 2 + N \cdot (4 + T \cdot (2 + 7 + 4 + \frac{M}{2} \cdot (4 + 28))) = \\
			= 2 + 4N + 13NT + \frac{32NTM}{2}  = 2 + 4N + 13NT + 16NTM
		\end{gathered}
	\end{equation}
	\item Цикл, который дополнительно нужен для подсчёта значений при нечётном размере любой из сторон матрицы, трудоемкость которого высчитывается по формуле:
	\begin{equation}
		\label{сomplexity:v_check}
		\begin{gathered}
			f_{check} = 3 +
			\begin{cases}
				0, & \text{чётная}, \\
				2 + N \cdot (4 + T \cdot (2 + 14)), & \text{иначе}.
			\end{cases}
		\end{gathered}
	\end{equation}
\end{enumerate}

Тогда для худшего случая (нечётный размер любой стороны матриц) имеем формулу:
\begin{equation}
	\label{сomplexity:vinograd_worst}
	\begin{gathered}
		f_{worst} = f_{init} + f_{atmp} + f_{btmp} + f_{cycle} + f_{check} \approx 16NMT = O(N^3).
	\end{gathered}
\end{equation}

Для лучшего случая (чётный размер любой стороны матриц) имеем формулу:
\begin{equation}
	\label{сomplexity:vinograd_best}
	\begin{gathered}
		f_{best} = f_{init} + f_{atmp} + f_{btmp} + f_{cycle} + f_{check} \approx 16NMT = O(N^3).
	\end{gathered}
\end{equation}

\subsection{Оптимизированный алгоритм Винограда}

Чтобы вычислить трудоемкость оптимизированного алгоритма Винограда, необходимо рассчитать трудоемкость его частей.
\begin{enumerate}
	\item Кэширование значения $\frac{M}{2}$ в циклах, которое равно 3.
	\item Создание и инициализация массивов $a_{tmp}$ и $b_{tmp}$ по формуле (\ref{сomplexity:v_init}).
	\item Заполнение массива $a_{tmp}$, трудоёмкость которого вычислена в формуле (\ref{сomplexity:v_atmp});
	\item Заполнение массива $b_{tmp}$, трудоёмкость которого вычислена в формуле (\ref{сomplexity:v_btmp});
	\item Цикл заполнения для чётных размеров, трудоёмкость которого считается по следующей формуле:
	\begin{equation}
		\label{сomplexity:v_opt_cycle}
		\begin{aligned}
			f_{cycle} &= 2 + N \cdot (4 + T \cdot (4 + 7 + \frac{M}{2} \cdot (2 + 10 + 5 + 2 + 4))) = \\
			&= 2 + 4N + 11NT + 11,5 \cdot NTM.
		\end{aligned}
	\end{equation}
	\item Условие, нужное для дополнительных вычислений при нечётном размере матрицы, трудоемкость которого считается по следующей формуле:
	\begin{equation}
		\label{сomplexity:v_opt_check}
		\begin{aligned}
			f_{check} = 3 +
			\begin{cases}
				0, & \text{чётная}, \\
				2 + N \cdot (4 + T \cdot (2 + 10)), & \text{иначе}.
			\end{cases}
		\end{aligned}
	\end{equation}
\end{enumerate}

Тогда для худшего случая (нечётный общий размер матриц) имеем формулу:
\begin{equation}
	\label{сomplexity:vinograd_opt_worst}
	\begin{aligned}
		f_{worst} &= 3 + f_{init} + f_{atmp} + f_{btmp} + f_{cycle} + f_{check} \approx\\&\approx 11NMT = O(N^3).
	\end{aligned}
\end{equation}

Для лучшего случая (чётный общий размер матриц) имеем формулу:
\begin{equation}
	\label{сomplexity:vinograd_opt_best}
	\begin{aligned}
		f_{best} = 3 + f_{init} + f_{atmp} + f_{btmp} + f_{cycle} + f_{check} \approx 11NMT = O(N^3).
	\end{aligned}
\end{equation}

\section*{Вывод}

В данном разделе были построены схемы алгоритмов умножения матриц, рассматриваемых в лабораторной работе, были описаны классы эквивалентности для функционального тестирования и модули программы.
Проведённая теоретическая оценка трудоемкости алгоритмов показала, что трудоёмкость алгоритма Винограда в случае оптимизации в 1.2 раза меньше, чем у простого алгоритма Винограда.
