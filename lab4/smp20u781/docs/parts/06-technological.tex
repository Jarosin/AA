\chapter{Технологическая часть}

В данном разделе приведены средства реализации программного обеспечения и листинги кода реализаций алгоритма.

\section{Средства реализации}

В качестве языка программирования для реализации данной лабораторной работы был выбран язык $C++$~\cite{cpp-lang}.
Данный выбор обусловлен наличием у языка встроенной библиотеки измерения количества тиков процессора, прошедших с момента запуска, и соответствием с выдвинутыми требованиям.

Для измерения времени использовалась функция \textit{clock()} из библиотеки \textit{ctime}~\cite{cpp-lang-time}.

\clearpage

\section{Сведения о файлах программы}

Данная программа разбита на следующие файлы:

\begin{itemize}[label=---]
	\item \texttt{main.cpp} --- файл, содержащий точку входа в программу, в нем происходит взаимодействие с пользователем и вызов алгоритмов;
	\item \texttt{src/threads.cpp} --- файл содержит функции, реализующие алгоритм поиска подстроки в файле;
	\item \texttt{src/file.cpp} --- файл содержит функции для работы с файлами;
	\item \texttt{src/cpu\_time.cpp} --- файл содержит функции, замеряющие время работы написанных функций.
\end{itemize}

В листингах \ref{lst:single} -- \ref{lst:parse} приведены реализации алгоритма поиска подстроки в файле.

\clearpage

\begin{lstlisting}[label=lst:single,caption=Функция поиска подстроки в файле с одним потоком.]
int seq_substring_replace(std::string &in_file, std::string &out_file, std::string &substr)
{
	std::ifstream in(in_file, std::ios::in);
	if (!in.is_open())
	{
		return 1;
	}
	std::ofstream out(out_file);
	if (!out.is_open())
	{
		return 1;
	}

	int total = find_file_line_total(in);
	if (substr_search_pos(in, out, substr, 0, total))
	{
		in.close();
		out.close();
		return 1;
	}

	in.close();
	out.close();
	return 0;
}
\end{lstlisting}

\begin{lstlisting}[label=lst:main_thread,caption=Функция работы основного потока при многопоточной реализации]
int thread_substring_replace(std::string &in_file, std::string &out_file, std::string &substr, size_t thread_total)
{
	reader_done = false;
	parsers_done = false;
	std::ifstream in(in_file, std::ios::in);
	if (!in.is_open())
	{
		return 1;
	}
	std::ofstream out(out_file);
	if (!out.is_open())
	{
		return 1;
	}
	std::string line;
	ThreadQueue<std::string> out_queue;
	ThreadQueue<std::pair<std::string, int>> in_queue;
	std::vector<std::thread> threads;
	std::thread thread_read(read_thread, std::ref(in), std::ref(in_queue));
	threads.push_back(std::move(thread_read));
	for (int i = 0; i < thread_total; i++)
	{
		std::thread thread_obj(thread_substr, std::ref(substr), std::ref(out_queue), std::ref(in_queue));
		threads.push_back(std::move(thread_obj));
	}
	std::thread thread_write(write_thread, std::ref(out), std::ref(out_queue));
	for (int i = 0; i < threads.size(); i++)
	{
		threads[i].join();
	}
	{
		std::lock_guard<std::mutex> _(parsers_done_mutex);
		parsers_done = true;
	}
	thread_write.join();
	in.close();
	out.close();
	return 0;
}
\end{lstlisting}

\begin{lstlisting}[label=lst:read,caption=Функция работы потока-читателя]
void read_thread(std::ifstream &in, ThreadQueue<std::pair<std::string, int>> &out_queue)
{
	if (!in.is_open())
	{
		return;
	}

	std::string line;
	int cur = 0;
	while (getline(in, line))
	{
		out_queue.push(std::pair<std::string, int>(line, cur));
		cur++;
	}
	std::unique_lock<std::shared_mutex> r(reader_done_mutex);
	reader_done = true;
	return;
}
\end{lstlisting}

\begin{lstlisting}[label=lst:write,caption=Функция работы потока-писателя]
void write_thread(std::ofstream &out, ThreadQueue<std::string> &in_queue)
{
	if (!out.is_open())
	{
		return;
	}

	std::string line;
	while (true)
	{
		if (in_queue.empty())
		{
			std::lock_guard<std::mutex> _(parsers_done_mutex);
			if (parsers_done)
				return;
			continue;
		}
		line = in_queue.pop();
		out << line;

	}
	return;
}
\end{lstlisting}

\begin{lstlisting}[label=lst:parse,caption=Функция работы потока, обрабатывающего строки]
void thread_substr(std::string &substr, ThreadQueue<std::string> &out_queue, ThreadQueue<std::pair<std::string, int>> &in_queue)
{
	size_t pos;
	std::pair<std::string, int> top;

	while (true)
	{
		try
		{
			top = in_queue.pop();
		}
		catch (...)
		{
			std::shared_lock<std::shared_mutex> r(reader_done_mutex);
			if (reader_done)
			{
				return;
			}
			continue;
		}

		pos = top.first.find(substr);
		while (pos != std::string::npos)
		{
			std::ostringstream stream;
			stream << top.second + 1 << " " << pos << "\n";
			out_queue.push(stream.str());

			pos = top.first.find(substr, pos + 1);
		}
	}
}
\end{lstlisting}

\section*{Вывод}

В данном разделе были рассмотрены средства реализации, а также представлен листинг реализации алгоритма поиска подстроки в файле.
