\chapter*{Заключение}
\addcontentsline{toc}{chapter}{Заключение}

В ходе выполнения лабораторной работы было выявлено, что многопоточная реализация алгоритма поиска подстроки медленнее, чем однопоточная.
Увеличение количества потоков может как уменьшить время выполнения программы, так и увеличить.

Цель, поставленная в начале работы, была достигнута: были получены навыки организации параллельного выполнения операций.
Были достигнуты все поставленные задачи.
\begin{enumerate}
	\item Описаны основы распараллеливания вычислений.
	\item Разработано программное обеспечение, которое реализует однопоточный алгоритм нахождения всех вхождений подстроки в файле.
	\item Разработана и реализована многопоточная версия данного алгоритма.
	\item Определены средства программной реализации.
	\item Выполнены замеры реального времени работы реализаций алгоритма.
	\item Проведено сравнение по времени работы реализаций алгоритма.
\end{enumerate}

При значении потоков превышающем число логических ядер (более 8 для устройства, на котором проводилось тестирование), затраты на содержание потоков превышают преимущество от использования многопоточности и время выполнения по сравнению с лучшим результатом (для 8 потоков) растут.
