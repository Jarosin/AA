\chapter{Аналитическая часть}

\section{Графовые модели программы}

Программа представлена в виде графа: набор вершин и множество соединяющих их направленных дуг.

\begin{enumerate}
    \item \textbf{Вершины}: процедуры, циклы, линейные участки, операторы, итерации циклов, срабатывание операторов и т. д.
    \item \textbf{Дуги} отражают связь (отношение между вершинами).
\end{enumerate}

Выделяют 2 типа отношений:
\begin{enumerate}
    \item операционное отношение~--- по передаче управления;
    \item информационное отношение~--- по передаче данных.
\end{enumerate}

Граф управления --- это граф, в котором вершины обозначают операторы, а дуги~--- операционные отношения.

Информационный граф --- это граф, в котором вершины обозначают операторы, а дуги~--- информационные отношения.

Операционная история --- это граф, в котором вершины обозначают срабатывание операторов, а дуги~--- операционные отношения.

Информационная история --- это граф, в котором вершины обозначают срабатывание операторов, а дуги~--- информационные отношения.
