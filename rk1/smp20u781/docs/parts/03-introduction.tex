\chapter*{Введение}
\addcontentsline{toc}{chapter}{Введение}

По мере развития вычислительных систем программисты столкнулись с необходимостью производить параллельную обработку данных для улучшения отзывчивости системы и ускорения производимых вычислений.
Благодаря развитию процессоров стало возможным использовать один процессор для выполнения нескольких параллельных операций, что дало начало термину "Многопоточность".

Целью данного рубежного контроля является написание ПО, выполняющего поиск всех вхождений подстроки в файл с учетом возможных опечаток, возможное количество которых зависит от длины строки.

Для поставленной цели необходимо выполнить следующие задачи.
\begin{enumerate}
	\item Описать основы распараллеливания вычислений и методы нахождения опечаток в строке.
	\item Разработать и реализовать многопоточную версию алгоритма поиск всех вхождений подстроки в файл с учетом возможных опечаток.
	\item Определить средства программной реализации.
\end{enumerate}
