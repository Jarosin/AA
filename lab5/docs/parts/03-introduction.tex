\chapter*{Введение}
\addcontentsline{toc}{chapter}{Введение}

По мере развития вычислительных систем программисты столкнулись с необходимостью производить параллельную обработку данных для улучшения отзывчивости системы и ускорения производимых вычислений.
Благодаря развитию процессоров стало возможным использовать один процессор для выполнения нескольких параллельных операций, что дало начало термину <<Многопоточность>>.

Целью данной лабораторной работы является получение навыков организации параллельного выполнения операций.

Для поставленной цели необходимо выполнить следующие задачи.
\begin{enumerate}
	\item Описать основы распараллеливания вычислений.
	\item Разработать программное обеспечение, которое реализует однопоточный алгоритм нахождения всех вхождений подстроки в файле.
	\item Разработать и реализовать многопоточную версию данного алгоритма.
	\item Определить средства программной реализации.
	\item Выполнить замеры реального времени, затрачиваемого на работу реализации алгоритма.
	\item Сравнить по времени работы версии реализации алгоритма.
\end{enumerate}
