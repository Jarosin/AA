\chapter{Аналитическая часть}
В данном разделе будет представлена информация о многопоточности и реализуемом алгоритме нахождения всех вхождений подстроки в файле.

\section{Многопоточность}

\textbf{Многопоточность} \text{(англ. \textit{multithreading})} --- это способность центрального процессора (ЦП) обеспечивать одновременное выполнение нескольких потоков в рамках использования ресурсов одного процессора~\cite{multithreading}.
Поток --- последовательность инструкций, которая может исполняться параллельно с другими потоками одного и того же породившего их процесса.

Процессом называют программу в ходе своего выполнения~\cite{process}.
Таким образом, когда запускается программа, создается процесс.
Один процесс может состоять из одного и более потоков.
Таким образом, поток является сегментом процесса, сущностью, которая выполняет задачи, стоящие перед исполняемым приложением.
Процесс завершается, когда все его потоки заканчивают работу.
Каждый поток в операционной системе является задачей, которую должен выполнить процессор.

При написании программы, использующей несколько потоков, следует учесть, что не всегда есть выигрыш от использования нескольких потоков.
Необходимо создавать потоки для независимых по данным и выполнять их параллельно, тем самым сокращая общее время выполнения процесса.

Одной из проблем, появляющихся при использовании потоков, является проблема совместного доступа к информации.
Фундаментальным ограничением является запрет на запись из двух и более потоков в одну ячейку памяти одновременно.

Из этого следует, что необходим примитив синхронизации обращения к данным -- так называемый мьютекс \text{(англ. \textit{mutex --- mutual exclusion})}.
Он может быть захвачен для работы в монопольном режиме или освобожден, чтобы позволить следующему потокому его захватить.
Так, если 2 потока одновременно пытаются захватить мьютекс, успевает только один, а другой будет ждать освобождения.

Набор инструкций, выполняемых между захватом и освобождением мьютекса, называется \textit{критической секцией}.
Поскольку в то время, пока мьютекс захвачен, остальные потоки, требующие выполнения критической секции для доступа к одним и тем же данным, ждут освобождения мьютекса для его последующего захвата, требуется разрабатывать программное обеспечение таким образом, чтобы критическая секция была минимальной по объему.

\section{Поиск подстроки}
Поиск подстроки в строке — задача поиска информации.
Применяется в виде встроенной функции в текстовых редакторах, СУБД, поисковых машинах, языках программирования и т.д.

В данной лабораторной работе проводится распараллеливание алгоритма поиска всех вхождений искомой подстроки в исходный файл с последующей записью номеров строки и столбца первого символа найденного вхождения в результирующий файл.

Так как последовательное чтение из файла быстрее, чем случайное, в параллельной реализации алгоритма будет использоваться 1 читающий поток ~\cite{multithreading}.
Кроме того, параллельное чтение с жесткого диска возможно только при наличии нескольких головок для чтения, что на практике далеко не всегда доступно.

Считанные строки будут равномерно распределены между несколькими потоками, так как их обработка не зависит от других строк.
Именно за счет распараллеливания обработки строк возможно получить какой-то выигрыш в данном алгоритме.

Поскольку одновременная запись в файл из множества потоков невозможна, для записи в файл будет также использоваться только 1 поток.

\section*{Вывод}
В данном разделе была представлена информация о многопоточности и исследуемом алгоритме.
