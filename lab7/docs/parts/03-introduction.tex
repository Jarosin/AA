\chapter*{Введение}
\addcontentsline{toc}{chapter}{Введение}

Каждый день количество информации в цифровом виде в мире увеличивается, что приводит к необходимости разрабатывать новые способы её хранения.
Любой новый способ хранения информации должен иметь алгоритм для её извлечения --- алгоритм поиска.

Алгоритм поиска --- это алгоритм, предназначенный для решения задачи поиска.
Задача поиска --- это задача извлечения информации из некоторой структуры данных.

Разные структуры данных используют разные алгоритмы поиска, в данной работе будут рассматриваться алгоритмы поиска целого числа в двоичном дереве поиска несбалансированном и сбалансированном (в АВЛ-дереве).

Целью данной лабораторной работы является сравнение алгоритмов поиска в сбалансированном и несбалансированном двоичном дереве поиска.

Для поставленной цели необходимо выполнить следующие задачи.
\begin{enumerate}
	\item Описать сбалансированное и несбалансированное двоичные деревья поиска.
	\item Описать алгоритмы поиска в сбалансированном и несбалансированном дереве поиска.
	\item Привести псевдокоды для алгоритма поиска в сбалансированном и несбалансированном деревьях поиска.
	\item Определить средства программной реализации.
	\item Реализовать разработанные алгоритмы.
	\item Выполнить замеры количества сравнений, необходимого для решения задачи поиска элементов в лучшем случае и в худшем случае для выполненных реализаций.
	\item Описать и обосновать полученные результаты в отчете о выполненной лабораторной работе.
\end{enumerate}
