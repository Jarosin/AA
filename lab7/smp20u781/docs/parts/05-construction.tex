\chapter{Конструкторская часть}

В данном разделе будут представлены листинги псевдокода для алгоритма поиска в сбалансированном и несбалансированном двоичном дереве поиска.

\section{Требования к ПО}

К программе предъявлен ряд требований:

\begin{itemize}[label=---]
	\item программа должна предоставить пользователю возможность выбрать сбалансированное или несбалансированное бинарное дерево поиска;
	\item программа должна дать пользователю возможность заполнить выбранное дерево значениями;
	\item программа должна дать пользователю возможность провести поиск значения в выбранном бинарном дереве;
	\item программа должна дать пользователю возможность замерить количество сравнений, необходимое для поиска значения в дереве поиска.
\end{itemize}

\section{Описание используемых типов данных}
При реализации алгоритмов будут использованы следующие структуры данных:
\begin{itemize}[label=---]
	\item узел дерева --- содержит значение и указатель на левого и правого потомка;
	\item бинарное дерево --- содержит корневой узел.
\end{itemize}

\section{Разработка алгоритмов}

Псевдокод для поиска в сбалансированном и несбалансированном бинарном дереве поиска представлен в листинге \ref{lst:pseudo}.
\begin{lstlisting}[label=lst:pseudo,caption=Псевдокод для поиска в бинарном дереве поиска]
	Входные данные: узел root - корень дерева,
					key - искомое значение
	Выходные данные: node - узел, содержащий искомое значение

	node = root
	while(True):
		if node is None or node.key == key then
			return node
		end if

		if node.key > key then
			node = node.left
		else
			node = node.right
		end if
	end while
	\end{lstlisting}

\section{Оценка количества сравнений}
Для дальнейших замеров количества сравнений необходимо определить, что является лучшим и худшим случаем для разработанного алгоритма.

Лучшим случаем является нахождение числа в корне дерева.
В таком случае потребуется 1 сравнение для нахождения искомого числа.

Худшим случаем в данной реализации будет являться отсутствие узла в дереве, так как в таком случае возможно будет выполнить $h + 1$ сравнений, где $h$ --- высота дерева, в то время как при нахождении искомого элемента на максимальной высоте дерева количество сравнений равно $h$.

\section*{Вывод}

В данном разделе были представлены листинги псевдокода для алгоритма поиска в сбалансированном и несбалансированном двоичном дереве поиска.
