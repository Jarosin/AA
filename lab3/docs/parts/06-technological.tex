\chapter{Технологическая часть}

В данном разделе приведены средства реализации программного обеспечения, листинги кода и функциональные тесты.

\section{Средства реализации}

В качестве языка программирования для реализации данной лабораторной работы был выбран язык $C++$~\cite{cpp-lang}.
Данный выбор обусловлен наличием у языка встроенной библиотеки измерения процессорного времени и соответствием с выдвинутыми требованиям.

Для измерения времени использовалась функция \textit{clock()} из библиотеки \textit{ctime}~\cite{cpp-lang-time}.

\clearpage

\section{Сведения о файлах программы}

Данная программа разбита на следующие файлы:

\begin{itemize}[label=---]
	\item \texttt{main.cpp} --- файл, содержащий точку входа в программу, в нем происходи общение с пользователем и вызов алгоритмов;
	\item \texttt{src/algs.cpp} --- файл содержит алгоритмы сортировка;
	\item \texttt{src/io.cpp} --- файл содержит функции для ввода и вывода массива;
	\item \texttt{src/cpu\_time.cpp} --- файл содержит функции, замеряющее процессорное время методов сортировки.
\end{itemize}

В листингах \ref{lst:insert} -- \ref{lst:bitonic} реализованы алгоритмы сортировки, а именно сортировка вставками, пирамидальная сортировка и битонная сортировка.

\clearpage

\begin{lstlisting}[label=lst:insert,caption=Функция сортировки вставками]
int insert_sort(int *array, int len)
{
	for (int i = 1; i < len; i++)
	{
		for (int j = i; j > 0 && array[j - 1] > array[j]; j--)
		{
			int tmp = array[j - 1];
			array[j - 1] = array[j];
			array[j] = tmp;
		}
	}
	return 0;
}
\end{lstlisting}

\clearpage

\begin{lstlisting}[label=lst:heap,caption=Функция пирамидальной сортировки]
void heapify(int *array, int i, int n)
{
	int largest;
	int done = 0;
	int l, r;

	while((2 * i <= n) && !done)
	{
		l = 2 * i;
		r = 2 * i + 1;

		if (l == n)
		{
			largest = l;
		} else if (array[l] > array[r])
			largest = l;
		else
			largest = r;

		if (array[i] < array[largest])
		{
			int temp = array[i];
			array[i] = array[largest];
			array[largest] = temp;
			i = largest;
		}
		else
			done = 1;
	}
}

void heap_sort(int *array, int n)
{
	if (array != nullptr && n > 0)
	{
		for (int i = n / 2; i >= 0; i--)
		{
			heapify(array, i, n - 1);
		}

		for (int i = n - 1; i >= 1; i--)
		{
			int temp = array[0];
			array[0] = array[i];
			array[i] = temp;
			heapify(array, 0, i - 1);
		}
	}
}
\end{lstlisting}

\clearpage

\begin{lstlisting}[label=lst:bitonic,caption=Функция битонной сортировки]
void compAndSwap(int *array, int first_index, int second_index, int dir)
{
	int temp;
	if (dir == (array[first_index] > array[second_index]))
	{
		temp = array[first_index];
		array[first_index] = array[second_index];
		array[second_index] = temp;
	}
}

void bitonicMerge(int *array, int start, int len, int dir)
{
	if (len > 1)
	{
		int half_len = len / 2;
		for (int i = start; i < start + half_len; i++)
			compAndSwap(array, i, i + half_len, dir);
		bitonicMerge(array, start, half_len, dir);
		bitonicMerge(array, start + half_len, half_len, dir);
	}
}

void bitonicSort(int *array, int start, int len, int dir)
{
	if (len > 1)
	{
		int half_len = len / 2;

		bitonicSort(array, start, half_len, 1);

		bitonicSort(array, start + half_len, half_len, 0);

		bitonicMerge(array, start, len, dir);
	}
}

void bitonic_sort(int *array, int length, int up)
{
	bitonicSort(array, 0, length, up);
}
\end{lstlisting}

\clearpage
\section{Функциональные тесты}

В таблице \ref{tbl:func_tests} приведены функциональные тесты для разработанных алгоритмов сортировки.
Все тесты пройдены успешно.
Под обозначением «$-$»\space в строках таблицы значит, что данный алгоритм сортировки не может работать с числом элементов, не являющимся степенью двойки, что истинно для битонной сортировки.

\begin{table}[ht]
	\small
	\begin{center}
		\begin{threeparttable}
		\caption{Функциональные тесты}
		\label{tbl:func_tests}
		\begin{tabular}{|c|c|c|c|c|}
			\hline
			\bfseries Массив
			& \bfseries Размер
			& \bfseries Ожидаемый р-т
			& \multicolumn{2}{c|}{\bfseries Фактический результат} \\ \cline{4-5}
			& & & \bfseries Вст./Пирам. & \bfseries Бит. \\
			\hline
			41 67 34 0 & 4 & 0 34 41 67 & 0 34 41 67 & 0 34 41 67 \\
			\hline
			31 57 24 -10 & 4 & -10 24 31 57 & -10 24 31 57 & -10 24 31 57 \\
			\hline
			1 2 3 4 & 4 & 1 2 3 4 & 1 2 3 4 & 1 2 3 4 \\
			\hline
			100 88 76 65 & 4 & 65 76 88 100 & 65 76 88 100 & 65 76 88 100 \\
			\hline
			-59 -33 -66 -100 -31 & 5 & -100 -66 -59 -33 -31 & -100 -66 -59 -33 -31 & - \\
			\hline
		\end{tabular}
		\end{threeparttable}
	\end{center}
\end{table}

\section*{Вывод}

Были реализованы алгоритмы: сортировка вставками, пирамидальная сортировка и битонная сортировка.
Проведено тестирование разработанных алгоритмов.
